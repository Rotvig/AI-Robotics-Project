\chapter{Discussion}
\label{chp:disc}
In this chapter the results from chapter \ref{chp:res} and the project work in general will be discussed and evaluated.

\section{Field test}
The lego robot was tested by placing the robot at a random start position in a known world. Then the lego robot uses it's sonar sensor in collaboration with the particle filter to localize itself very efficient. The robot locates itself, calculates a path and steers itself to the goal at coordinate (25,25). 

\section{Hardware problems}
Using the LEGO Mindstorms EV3 setup for the robot was a good decision in the sense that it was easy and quick to get started, but have also given some problems throughout the project.

\begin{itemize}
	\item The infrared sensor has proven to be too unstable to use in the field test.
	The sensor gives good results in the unit tests, but the fact that the sensor relays on reflected light, means that the distance measurements is effected by shifting light conditions and different materials.
	The field test has therefore been conducted only using the sonar sensor.
	\item The motors rotation speed and toque are very effected by the amount of change on the batteries.
	This is a problem in relation to the PID parameters, that is defined by testing the robot at a given battery charge.
	When the robot moves and start to drain power over time, the motor characteristics change and the PID parameters starts to regulate more poorly.
	This has some how been fixed by defining the PID parameters after a fully charged battery, and then always have at least half the battery charged.
	\item The communication to the robot is done using Bluetooth, this is a fine solution when working.
	In cases where the code enters an exception, or by other means doesn't get the Bluetooth connection terminated correctly, the robot can get in a state where it is no longer possible establish a connection, and the robot requires a restart.
	It is time consuming during test, to have to restart the robot all the time.
\end{itemize}

\section{Improvements and comments}

To improve the lego robot and the debugging capabilities, some functionalities should be added or changed.

One of those is the particle filters noise simulation (KIM SKRIV HER).

An improvement which can be added to the particle filter is the estimation of the robot's position. At the moment the position is calculated by the using the mean position of the particles. Instead of doing this clustering can be used, to achieve a more accurate estimation of the lego robot's position.

A functionality which will help the debugging of the robot is plotting the route of the robot’s path, from the robot’s position to the goal. This can help to visualize the path the robot intends to use.

To help the robot avoiding to drive into objects, the objects is theoretical enlarged. Instead of increasing the size of the objects, the lego robot itself could have a size.
At the moment, only squares are supported in the world map. This can be improved by implementing other types of geometrical shapes.

To control the robot’s rotation a PID-Controller is used together with a gyroscope. The PID gains can be optimized better to get a high-rise time, lower overshoot and a faster settling time.

Furthermore the PID-Controller only adjusts the rotation of the lego robot, but it can also be added to the forward and backward motion. To achieve a more correct path.

The lego robot’s design can be improved by adding an motor to the sonar sensor, so it rotates around itself, instead of having the whole robot rotating around itself. 

Furthermore an LIDAR can be built instead of using a sonar, to improve the accuracy of the measurement.