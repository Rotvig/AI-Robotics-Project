\chapter{SLAM}
\label{chp:slam}

This chapter describes the basic concepts behind the robotics problem known as \emph{Simultaneous Localization And Mapping (SLAM)}, which addresses the issues of when a robot doesn't know the map of its environment.
The principle behind SLAM is for a robot to use measurements and control metrics to construct a map, while simultaneously localizing it self relative to this map, as it moves within the given environment.

SLAM is currently a substantial topic of research within robotics, with a lot of advanced methods, because the robot might loose track of where it is by virtue of its own motion uncertainties.
However this report will focus on Graph SLAM, an older version of SLAM that is easily comprehensible and serves as a good introduction to the concepts of SLAM in practice.\\\\

\noindent In Graph SLAM we can reduce the mapping problem to additions into a matrix and a vector, and a simple matrix multiplication.\\
The matrix, $\Omega$, and the vector, $\xi$, is generated by gathering the following constraints of the robot:
\begin{itemize}
    \item The initial location
    \item Relative motion
    \item Relative measurements to landmarks
\end{itemize}
The matrix and the vector can then be used find the best estimate, $\mu$ of both all the robot locations and all the landmark positions, by the formula:
\begin{align*}
    \mu = \Omega^{-1} \xi
\end{align*}
The best way to explain the process is through a 1D example.


\myFigure{slamEx1}{Illustration of Graph SLAM example. The triangles represent the robot at different times, the cylinders represents landmarks, the solid arrows represents the robots movements and the dashed arrow represents measurements.}{fig:slamEx1}{0.7}

The robot has an initial location which we define as $x_0 = 0$, and moves 5 steps forward to a new position, $x_1$.
This movement constraint can be described as:
\begin{align*}
x_1 &= x_0 + 5\\
-5  &= x_0 - x_1
\end{align*}

\noindent These equations, including the initial location equation, $x_0 = 0$,  are what should be added into the matrix $\Omega$ and the vector $\xi$, such that we get:
\begin{table}[!h]
    \centering
\begin{tabular}{cc|rrrrr c crc}
          &       & $x_0$ & $x_1$ & $x_2$ & $L_0$ & $L_1$ & 
          
          \qquad &        &                        & \\
          
          \cline{2-7}
          & $x_0$ &   2   &  -1   &  0    &  0    &   0   &  
          
                 &        &\multicolumn{1}{r|}{-5} & $x_0$\\
          
          & $x_1$ &  -1   &   1   &  0    &  0    &   0   &  
                 
                 &        & \multicolumn{1}{r|}{5} & $x_1$\\

$\Omega=$ & $x_2$ &   0   &   0   &  0    &  0    &   0   &  
                 
                 & $\xi=$ &\multicolumn{1}{r|}{0} & $x_2$\\
                 
          & $L_0$ &   0   &   0   &  0    &  0    &   0   &  
                 
                 &        & \multicolumn{1}{r|}{0} & $L_0$\\
                 
          & $L_1$ &   0   &   0   &  0    &  0    &   0   &  
          
                 &        & \multicolumn{1}{r|}{0} & $L_1$\\
\end{tabular}                                               
\end{table}

\noindent Now suppose that the robot moves backwards by 4 steps to $x_2$:
\begin{align*}
x_2 &= x_1 - 4\\
-4  &= x_2 - x_1
\end{align*}
This is now added to the already existing $\Omega$ and $\xi$, such that we get:
\begin{table}[!h]
    \centering
    \begin{tabular}{cc|rrrrr c crc}
        &       & $x_0$ & $x_1$ & $x_2$ & $L_0$ & $L_1$ & 
        
        \qquad &        &                        & \\
        
        \cline{2-7}
        & $x_0$ &   2   &  -1   &  0    &  0    &   0   &  
        
        &        &\multicolumn{1}{r|}{-5} & $x_0$\\
        
        & $x_1$ &  -1   &   2   &  -1    &  0    &   0   &  
        
        &        & \multicolumn{1}{r|}{9} & $x_1$\\
        
        $\Omega=$ & $x_2$ &   0   &   -1   &  1    &  0    &   0   &  
        
        & $\xi=$ &\multicolumn{1}{r|}{-4} & $x_2$\\
        
        & $L_0$ &   0   &   0   &  0    &  0    &   0   &  
        
        &        & \multicolumn{1}{r|}{0} & $L_0$\\
        
        & $L_1$ &   0   &   0   &  0    &  0    &   0   &  
        
        &        & \multicolumn{1}{r|}{0} & $L_1$\\
    \end{tabular}                                               
\end{table}

\noindent If the robot at position $x_1$ saw a landmark, $L_0$, at a distance of 9 away and at position $x_2$ saw a landmark, $L_1$, at a distance of -3 away,  this should be added to the matrix and vector as well.
\begin{align*}
x_1 - L_0 &= -9\\
9  &= L_0 - x_1\\
x_2 - L_1 &= 3\\
-3  &= L_1 - x_2
\end{align*}

\newpage 
Giving us:
\begin{table}[!h]
    \centering
    \begin{tabular}{cc|rrrrr c crc}
        &       & $x_0$ & $x_1$ & $x_2$ & $L_0$ & $L_1$ & 
        
        \qquad &        &                        & \\
        
        \cline{2-7}
        & $x_0$ &   2   &  -1   &  0    &  0    &   0   &  
        
        &        &\multicolumn{1}{r|}{-5} & $x_0$\\
        
        & $x_1$ &  -1   &   3   &  -1    &  -1    &   0   &  
        
        &        & \multicolumn{1}{r|}{0} & $x_1$\\
        
$\Omega=$&$x_2$ &   0   &   -1   &  2    &  0    &  -1   &  
        
        & $\xi=$ &\multicolumn{1}{r|}{-1} & $x_2$\\
        
        & $L_0$ &   0   &   -1   &  0    &  1   &   0   &  
        
        &        & \multicolumn{1}{r|}{9} & $L_0$\\
        
        & $L_1$ &   0   &   0   &  -1    &  0    &   1  &  
        
        &        & \multicolumn{1}{r|}{-3} & $L_1$\\
    \end{tabular}                                               
\end{table}

\noindent If we now calculate the landmark and robot positions we get:
\begin{table}[!h]
    \centering
    \begin{tabular}{cr|c}
                                      &  0 & $x_0$ \\
                                      &  5 & $x_1$ \\
       $\mu = \Omega^-1 \cdot \xi = $ &  1 & $x_2$ \\
                                      & 14 & $L_0$ \\
                                      & -2 & $L_1$ \\
    \end{tabular}                                               
\end{table}

\noindent where we can see we get what we expected, for both the robots position at each time index, and the landmarks location.\\\\

\noindent Normally SLAM problem would be in at least 2 dimensions, which Graph SLAM also is able to handle, either by expanding the $\Omega$ matrix and the $\xi$ vector, or by creating separate matrices for each dimension.