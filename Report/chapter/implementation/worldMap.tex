
\section{World Map}
\label{sec:worldMap}

The world map is one of the basic requirements needed for both localization and path finding.
In this project the world map is used to do fictive particle measurements for the particle filter, and provide a bit map containing obstacles for the A* algorithm.
The world map class is containing the length and width of the world, and a list of objects in the world.
It has been decided to only use squares objects in the world map, and these are modelled as a set of 4 lines.
Each square object is defined by 4 parameters, a corner position, length, width and an orientation offset of the x axis in degrees.
A useful geometry library called \emph{MathNet.Spatial} is used to easy modulation of points, lines and angles, along easy access to geometry algebra as line length and the intersect point of two lines.

\subsection{Particle measurements}

The robot is measuring the distance to surrounding objects as the means of localization, and the particles therefore needs the ability to do the same action in the world map.
This is implemented by creating a line from the particle in the particle orientation direction, and calculate the intersect point with the world edge lines and object lines, as seen on figure \ref{fig:particleMeasurement}
The distance to the closest intersect point from the particle is chosen as the fictive measurement of that particle.

\myFigure{Implementation/WorldMap/particleMeasurement}{The principle of particle distance measurements in the world map.}{fig:particleMeasurement}{0.9}

\pagebreak