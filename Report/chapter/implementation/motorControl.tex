
\section{Motor Control}
\label{sec:motorControl}

The robot has two mounted NXT motors, one on either side.
The robot movement is controlled, by setting the power and direction of these two motors with the NXT functions seen in listing \ref{list:NXT_moveFuncs}.
The distance that the robot is travelling is a function of both the applied power and how long the action is lasting.
By experiments is has been defined that the robot is moving 2 cm per period of 250 Ms with a power of 25 applied to the motors.
The last parameter \emph{Brake} stops the motor after timeout, if set to \emph{True}.

\begin{lstlisting}[caption={The two NXT funtions used to control the motors.}, label=list:NXT_moveFuncs]
SetMotorPolarityAsync(OutputPort, Polarity);
TurnMotorAtPowerForTimeAsync(OutputPort, Power, MoveMs, Brake);
\end{lstlisting}

\subsection{PID controller}

It has been observed that the motors have slightly different torque, this have the effect that if the same power is applied to both motors, the robot will swing slightly to the right.
The best way to avoid this is to add a PID regulator, based on a gyroscope.
This way different powers can be applied to the two motors, depending on the error of gyroscope measurement.
A PID regulator also gives the ability of making precise turns.

\myFigure{Implementation/MotorControl/PID_TurnFigure}{Principle of PID regulator turning robot as function of gyroscope error.}{fig:PID_TurnFigure}{0.55}

Figure \ref{fig:PID_TurnFigure} show the principle of the implemented PID regulator.
The \emph{error} is a direct measure of how much the robot is of course in degrees. A positivist gyroscope value is a result of the robot turning left.

$$ error = gyroValue - gyroTargetValue $$

The \emph{turnPower} value is specifying the action to be done depending on the \emph{error}. This is where the PID regulator is added:

$$ integral = integral + error $$
$$ derivative = error - lastError $$
$$ turnPower = Kp * error + Ki * integral + Kd * derivative $$

The \emph{turnPower} value is know applied to the two motors to minimize the error.
A positive \emph{turnPower} means that the robot is turning left, and more power is therefore applied to the left motor, and vice versa for the right motor.

$$ powerRight = movePower - turnPower $$
$$ powerLeft = movePower + turnPower $$

The \emph{movePower} parameter is defining the basic operation.
This value is 25 for forward motion, -25 for backwards motion, and 0 for rotation.

\subsection{PID Tuning}

The PID parameters are tuned in a scenario of preforming a 90 degree turn, following the Ziegler–Nichols method.
This method starts by finding the critical gain \emph{$K_c$}, by setting \emph{$Ki$} and \emph{$Kd$} to 0, and increase \emph{$Kp$} until oscillation is just stable.
Figure \ref{fig:PID_OscillationGraf} show the oscillation state of the system with \emph{$Kp = 2.4$}.

\myFigure{Implementation/MotorControl/PID_OscillationGraf}{Graph of system with  ultimate gain: \emph{$K_c = 2.4$}}{fig:PID_OscillationGraf}{1}

From figure \ref{fig:PID_OscillationGraf}, is it possible to determined the oscillation period \emph{$P_c$}.
This is done by measure the start and end time of 10 oscillation, calculate the mean period and multiply by the motion time step of 250 ms.

$$ P_c = \frac{67.6518-8.524}{10} * 0.250 = 1.478 $$

The loop time is also needed for the Ziegler–Nichols method.

$$ dT = 0.250 $$

\myFigure{Implementation/MotorControl/PID_Params}{Table of PID constants when using the Ziegler–Nichols method.}{fig:PID_Params}{0.7}

\emph{$K_c$}, \emph{$P_c$} and \emph{$dT$} is used to calculate the PID parameters following the table in figure \ref{fig:PID_Params}\footnote{\url{http://www.inpharmix.com/jps/PID_Controller_For_Lego_Mindstorms_Robots.html}}.

$$ Kp = 0.6 * K_c = 1.44 $$

$$ Ki = \frac{2*Kp*dT}{P_c} = 0.487 $$

$$ Ki = \frac{Kp*P_c}{8*dT} = 1.064 $$

These 3 PID parameters are implemented in the software and tested in two movement cases; a 90 degree left turn seen in figure \ref{fig:PID_90DegreeTurn}, and a forward movement for 30 cm seen in figure \ref{fig:PID_Forward}.

\todo{Make figure clearing}

\mySubFigure{Implementation/MotorControl/PID_90DegreeTurnError}{Implementation/MotorControl/PID_90DegreeTurnMotors}
{Graphs of PID regulation of a 90 degree turn. The \emph{gyroTargetValue} is set to 90 for the robot to do a 90 degree left turn.}
{Error plot of the turn. The Error starts in -90, makes about 50\% overshot and swing in to a stable value in about 5 sec.}
{The power values to the two motors, a positive value means forward spin and a negative value means backwards spin. Green is the left motor and blue is the right motor. It can be seen that the values match with a left swing and that the values go towards 0 with deceasing error.}
{fig:PID_90DegreeTurn}{fig:PID_90DegreeTurnError}{fig:PID_90DegreeTurnMotors}

\mySubFigure{Implementation/MotorControl/PID_ForwardError}{Implementation/MotorControl/PID_ForwardMotors}
{Graphs of PID regulation of a forward motion of 30 cm. The \emph{gyroTargetValue} is set to 0 for the robot to do a forward motion.}
{Error plot of the forward motion. The Error starts in 0, starts to oscillate a bit do to unbalanced motors, and stabilize after 4-5 sec.}
{The power values to the two motors, a positive value means forward spin and a negative value means backwards spin. Green is the left motor and blue is the right motor. It can be seen that the values swing in correlation with the error, and that there is a little offset between the motors at the stable state.}
{fig:PID_Forward}{fig:PID_ForwardnError}{fig:PID_ForwardTurnMotors}

\FloatBarrier

The PID motion test seen in figure \ref{fig:PID_90DegreeTurn} and \ref{fig:PID_Forward}, gives fine results with the calculated PID parameters.
Both the turn and forward case have a bit too big overshoot, that can be fixed by fine tuning of the PID parameters a bit.
Decrease \emph{$Kp$}, \emph{$Ki$} or increase \emph{$Kd$}, all have a positive effect on the overshoot, but do also have a negative effect on the rise time and settling time, making the robot turn slower.
It comes down to find a good balance between high turn speed, low turn time and high turn precision.

\pagebreak