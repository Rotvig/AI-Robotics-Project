\chapter{Implementation}
\label{chp:impl}

\section{Particle Filter}
\section{World Map}
\section{A* search algorithm}

The A* search algorithm has been used in this project to find a route to the goal. 

The input to the algorithm is a matrix which consist of areas where  can go and not go. The matrix also contains the information of the robots' position, and a goal for the robot.

\begin{lstlisting}[caption={An example of the matrix which the A* uses as input.}, label=a_star]
char[,] matrix = new char[,] {{'-', 'S', '-', '-', 'X'},
							  {'-', 'X', '-', 'X', '-'},
							  {'-', '-', 'X', '-', 'X'},
							  {'X', '-', 'X', 'E', '-'},
							  {'-', '-', '-', '-', 'X'}};
\end{lstlisting}

In listing \ref{a_star} the 'X's marks places the robot can not move through. The 'S' is the position of the robot, and 'E' is the goal. The array itself is a multidimensional char array.

The nodes in the algorithm consist not only of coordinate x and y. The nodes also needs to contain a reference to its parent node (where it came from). This is used to backtrace the route. The node also needs to contain the information g(n) which is the cost of the path from the start node to node n. The node also contain h(n) which is a heuristic that estimates the cost of the cheapest path from n to the goal 'E'. The last property of the node is f(n) and f(n) = g(n) + h(n).

In the implementation a dictionary is used for the open and closed lists. Where the key is a string containing the coordinate x and y. For example the key for x = 1, y = 1 would be "11". The reason for this is that later on in the search the coordinate can then be checked if it is in the closed dictionary. If the coordinate is in the closed dictionary then the node will be ignored.



\section{Motor Control}

