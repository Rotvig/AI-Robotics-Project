\chapter{Conclusion}
\label{chp:conc}

In this project the theory from the Udacity course Artificial Intelligence for Robotics\footnote{\url{https://www.udacity.com/course/artificial-intelligence-for-robotics--cs373}} have been used to develop a self driving robot.
The course consisted of six main topics which all are elaborated in part \ref{part:one}, Theory.
The project consist of placing a LEGO robot at an unknown position in a known world, and then by own means localize it self before navigating around obstacles to a goal position.

The actual theory used to achieve a self driving robot are, particle filter, A*, PID-Controller and sensor fusion.

Everything was implemented with success, apart from the sensor fusion, due to the infrared sensor, that proved to be too unstable to use.
The decision was therefore made to exclude the infrared sensor and only use the sonar sensor.

The main project apart from some hardware challenges, have been a great success.
There are some improvement which could make the robot more efficient in different ways, but the overall goal of making a self driving robot is a archived.